\documentclass{article}

\usepackage{setup}

\newsource{\sAshikagaBukan}{AB}{Ashikagake Bukan (足利家武鑑; “House Ashikaga Book of Heraldry”). \url{http://archive.wul.waseda.ac.jp/kosho/bunko20/bunko20_00368/bunko20_00368_0007/bunko20_00368_0007.html}}
\newsource{\sDowerCrests}{Dower}{John W. Dower. \emph{The Elements of Japanese Design: A Handbook of Family Crests, Heraldry \& Symbolism}. Shambhala Publications, 1990.}
\newsource{\sEncyclopediaOfShinto}{EoS}{\emph{Encyclopedia of Shinto}. \url{http://k-amc.kokugakuin.ac.jp/DM/dbTop.do?class_name=col_eos}}
\newsource{\seMuseum}{eM}{e-Museum: National Treasures \& Important Cultural Properties of National Museums, Japan. \url{http://www.emuseum.jp/}}
\newsource{\sEricObershaw}{EO}{Obershaw, Eric. Photographs. \url{http://www.jcastle.info/}}
\newsource{\sFCoJ}{FCoJ}{\emph{Family Crests of Japan}. Stone Bridge Press, 2007.}
\newsource{\sKenmonShokamon}{KS}{\emph{Kenmon Shokamon} (見聞諸家紋; ``Various Observed Family Crests''). 1467--1470. \url{http://dl.ndl.go.jp/info:ndljp/pid/2533035}}
\newsource{\sKamonNoJiten}{KnJ}{Takasawa Hitoshi (高澤等). \emph{Kamon no Jiten} (家紋の事典; ``Family Crest Encyclopedia''). T\=oky\=o-d\=o Shuppan, 2008.}
\newsource{\sTurningPoint}{Oribe}{Miyeko Murase, Ed. \emph{Turning Point: Oribe and the Arts of Sixteenth-Century Japan}. The Metropolitan Museum of Art, New York, 2003.}
\newsource{\sOhatamoto}{OS}{\emph{Ohatamoto s\=oshirushizu} (御簱本惣印図; “Shogunal Vassals All Emblem Drawings”). Library of Congress. 1634. \url{http://fireflies.xavid.us/wp-content/uploads/2015/11/Ohatamoto-S\%C5\%8Dshirushizu.pdf}}
\newsource{\sOUmajirushi}{OU}{\emph{O-umajirushi} (御馬印; ``Honorable Battle Standards''). National Diet Library, Japan. 1624--1644. \url{http://dl.ndl.go.jp/info:ndljp/pid/1288484?__lang=en}}
\newsource{\sRijks}{R:N}{``Arrival of a Portuguese ship''. Rijksmuseum. \url{https://www.rijksmuseum.nl/en/collection/AK-RAK-1968-1-B}}
\newsource{\sRijksB}{R:B}{``Japans lakkistje''. Rijksmuseum. \url{https://www.rijksmuseum.nl/en/collection/NG-KOG-F6-5A,4}}
\newsource{\sRijksH}{R:H}{``Helm in de vorm van een hofhoed''. Rijksmuseum. \url{https://www.rijksmuseum.nl/en/collection/AK-RAK-2009-2,0}}
\newsource{\sSamuraiHeraldry}{SH}{Stephen Turnbull. \emph{Samurai Heraldry}. Osprey Publishing, 2002.}
\newsource{\sSamuraiSourcebook}{SS}{Stephen Turnbull. \emph{The Samurai Sourcebook}. Cassel \& Co, 2000.}
\newsource{\sCommons}{WC}{Wikimedia Commons. \url{https://commons.wikimedia.org/}}

% \TODO{Update Kenmon Shokamon page numers}
\begin{document}

\chapter{\Huge Heraldic Display in Medieval Japan}

\begin{center}
\large by Kih\=o (希鴎)
\end{center}

\section{What is a Mon?}

  Japan developed a system of identifying symbols or crests that is
  frequently compared to European heraldry.  These \textbf{crests are called
  `mon' (紋) or `monshō' (紋章)}. While there are some
  similarities to European devices, and both systems were used for
  similar identification purposes, the Japanese system has different
  basic principles and a different aesthetic.

  Mon are generally simple designs.  The vast majority
  of mon used a single color on a solid background.  In addition, the way
  different graphical elements were combined was limited; many mon
  were constructed of only one type of element, and those that
  combined multiple different types of `charge' did so in standardized
  ways, such as adding an enclosure around another element.

  What were mon used for?  \textbf{Primarily, they were used for
  identification}, and by extension for decoration.  Samurai would use
  mon on banners to identify unit affiliation.  They were used as
  designs on kimono, armor, camp curtains, roof tiles, carriages, and
  personal items.

  \textbf{The most common type of mon is the `kamon' (家紋), or
  ``family crest".}
  Kamon were used to identify family/clan membership or affiliation.
  However, there was never a one-to-one relationship between mon and
  families: different branches of a family could easily use distinct
  (sometimes related) mon, and multiple mon could be used by the same
  individual in different situations.  Furthermore, different families
  could share similar or identical mon, either by coincidence (for
  families in distant parts of Japan) or due to a strong clan granting
  its mon to a favored supporter.
  
  The use of mon was
  not restricted to families: non-family groups, such as temples and
  shrines, could also use mon.
  \textbf{Crests used by Shintō shrines are generally called `shinmon'
  (神紋), ``divine crests".}  Crests for shrines
  became increasingly popular
  during the \KamakuraPeriod.  Shrine crests were generally similar to
  family crests, and could be adopted in either direction: a family
  could adopt the crest of a shrine they had a strong connection to,
  or a shrine could use the crest of a patron family.  Crests could
  also move from shrine to shrine, for example when an enshrined \emph{kami}
  (Shint\=o deity) was transferred.  These sorts of changes could
  leave a shrine with more than one mon in some
  cases.\sEncyclopediaOfShinto{Shinmon} Buddhist temples shared
  many practices with Shintō shrines in period, and also used
  crests in a similar way.\sFCoJ{127}

\section{Display}

  The earliest known use of mon was not on banners, but on Imperial ox
  carts going back as far back as the \NaraPeriod.\sSamuraiHeraldry{6}
  From here mon spread to be used by the Imperial family more generally
  and by the \HeianPeriod{} this had spread to the court nobility, 
  though given the small and
  highly interconnected noble class, there was little need to use them
  for identification.\sDowerCrests{4}

  \textbf{Mon were not used by the samurai warrior class until the
  12th-century Gempei wars}, where they were used on wooden mantlet
  (``war-door") shields, camp curtains, and flags.\sSamuraiHeraldry{8}
   Later in period, their use became progressively more
  widespread, leading to their use in such places as armor, sword guards,
  and other pieces of
  warriors' equipment as well as boxes, and paper lanterns.\sDowerCrests{11}
  By the \MomoyamaPeriod, the usage of mon had been common enough to
  spread to merchants in the capital.\sTurningPoint{224}

  \begin{figure}[b!]
  \begin{subfigs}
    \jpgfigure[200]{Nine Stars Oxcart}{oxcart}{A shadowed nine stars mon on an oxcart, 13th c.\seMuseum{100148}}
    \jpgfigure[185]{Gentian Seat}{seat}{An abbot's seat with gentian mon, 12th c.\sCommons{:Raiban (Abbot's seat)}}
  \end{subfigs}
  \end{figure}


  Some mon motifs
  originate in fabric patterns, and mon designs were used on
  clothing from an early age without serving as personal symbols.  In the \MuromachiPeriod, some warrior
  families wore highly elaborate kimono called `daimon', or
  ``great crest", incorporating family crests half as tall as the
  wearer on large billowing sleeves.\sDowerCrests{12} By the late
  Muromachi period, family crests were standardly used in more
  restrained ways to decorate various \textbf{everyday and formal garments}
  worn by samurai, generally with three or five prominent 1.5-inch
  mon.\sDowerCrests{15}  Relatedly, mon could be used on \textbf{armor},
  either on small metal fastenings, as large metal crests on helmets,
  or as large designs on breastplates. They were also used on personal items,
  such as \textbf{dishes} and \textbf{boxes} (e.g., \emph{figure \ref{box}}).

  Another place where mon were displayed was on \textbf{noren}, cloth curtains used 
  in doorways to keep out dust and identify a business.  While modern 
  noren are generally labeled with kanji or decorative paintings, 
  \MomoyamaPeriod{} folding screen paintings show the use of 
  mon on noren.\sTurningPoint{224} (See \emph{figure \ref{noren}}.)
  A similar use of mon was on
  the camp curtains used in military encampments.

  Mon were also used on stone blocks for building castles. Since many
  samurai would contribute building materials from differen quarries,
  mon marks called kokuin were used to identify the stones' source.
  (See \emph{figure \ref{kokuin}}.)

  \begin{figure}[t]
  \begin{subfigs}
    \jpgfigure[90]{Wood Sorrel and Chrysanthemum Box}{box}{1650\sRijksB{}}
    \subfigure[90]{One and Rhombus}{noren}{1600\sRijks{}}
    \jpgfigure[90]{War Fan and Circles Kokuin}{kokuin}{A war fan and circles
      kokuin from Edo Castle\sEricObershaw{}}
  \end{subfigs}
  \end{figure}

  Mon were, of course, also used on several different varieties of
  \textbf{banners}.  Many different styles of banner were used on the
  battlefield over time. Distinctive banners carried by attendants or
  attached to the backs of armor were common for battlefield
  identification.

  It is interesting to note that mon were not the only form of
  heraldry used on the battlefield.  Banners could be solid-colored,
  have only colored stripes, or have more complicated pictures or
  passages of text.  In addition, daimyō, the samurai warlords who
  fought over \SengokuPeriod{} Japan, often used large three-dimensional
  objects as personal standards instead of banners.

\section{Regulation}

  While mon were regulated in Japan the way coats of arms were in
  Europe, this was a relatively late development, not occurring until
  the \EdoPeriod.  \textbf{For the most part, in the SCA period, mon were
  simply chosen by the bearer, and the mostly-consistent graphical
  style and rules were the result of custom}, not regulation.  What
  rolls of mon we have are descriptive, often recounting how mon were
  used in a particular battle, rather than a canonical assignment of
  mon to families.  This lack of registration is why samurai based in different
  parts of Japan could inadvertently develop similar mon.  (This at times
  caused trouble for travelers, and could lead to a visitor
  temporarily using a substitute crest to avoid giving
  offense.\sDowerCrests{14})

  \begin{figure}[b]
  \begin{subfigs}
  \subfigure{Bullseye Banner}{2c}{\sOUmajirushi{2.7}
  }
  \subfigure{Tricolor Sun Fan}{3c1}{Satake Yoshinobu, 1614\sSamuraiHeraldry{62}
  }
  \subfigure{Three Color Kanji Banner}{3c2}{\sOUmajirushi{3.26}
  }
  \subfigure{Three Color Diamond Banner}{3c3}{\sOUmajirushi{6.9}
  }
  \subfigure{Gold Banner}{solid}{\sOUmajirushi{1.13}
  }
  \subfigure{Gold Stripe Banner}{basic1}{\sOUmajirushi{1.24}
  }
  \subfigure{Indented Lines Banner}{basic2}{\sOUmajirushi{1.32}
  }
  \end{subfigs}
  \end{figure}

\section{Color}

  While color is a major part Western heraldry, mon have a
  stronger association with shape than with color.  Mon today are
  defined purely by their shape, ignoring color; however, color is an
  important part of period usage.

  Before the \EdoPeriod, mon were strongly associated with particular
  color schemes.  One main use of mon was for battlefield
  identification, and distinctive colors makes it easier to recognize
  banners from a distance.  While the same mon with different colors
  could be used by different members of the same family or different
  divisions of the same army,\sSamuraiHeraldry{24} samurai in the
  \SengokuPeriod{} used consistent color schemes for their banners.
  More decorative uses of mon, such as on clothing or lacquered items,
  would not necessarily use these distinctive colors, however.

  So, what colors were used?  Most commonly, the \textbf{five ``lucky
  colors": yellow, blue,
  red, white, and
  black}.\sSamuraiHeraldry{24} (Green is traditionally considered
  a shade of blue in Japan, though in some cases ``light blue'',
  ``blue/green'', and ``navy blue'' are distinguished.) There are also examples
  of purple, gold, silver, and light yellow as distinct from yellow.
  These colors can be found in a number of
  combinations, generally emphasizing contrast.

  \textbf{While most mon only used two colors}, a foreground and a background
  (e.g., \emph{figure \ref{2c}}), \textbf{there are a few examples of 3-color mon}
  used on banners (\emph{figures \ref{3c1}, \ref{3c2},
  \ref{3c3}}).\sSamuraiHeraldry{B2,H9} Since there was no central
  authority for mon, there was nothing except custom to enforce the
  two-color style, and some samurai did their own thing.

  It is worth noting that, on the battlefield, \textbf{banner color was used for identification in period more
  consistently than mon}.  It is unclear whether there were mon on the
  earliest documented Japanese uses of identifying banners, the
  12th-century Gempei War and the preceding rebellions, but it is well
  established that the opposing sides used red and white banners for
  identification.\sSamuraiHeraldry{9} Even in the late period
  Sengoku battles, many samurai used solid-color
  banners\sSamuraiHeraldry{43.3} (\emph{figure \ref{solid}}) or ones with basic graphical
  elements but no mon (\emph{figures \ref{basic1}, \ref{basic2}}).\sSamuraiHeraldry{J}

\clearpage

  \begin{figure}
  \begin{subfigs}
  \subfigure{16-petal Chrysanthemum}{chrys}{Chrysanthemum, Emperor Antoku, 1185\sSamuraiHeraldry{A 2}}
  \jpgfigure{Three 12-petal Chrysanthemums}{3chrys}{12-petal Chryrsanthemums, 1470\sKenmonShokamon{13}}
  \subfigure{Five-three paulownia}{paul}{Paulownia, 1392\sSamuraiHeraldry{6 2}
  }
  \subfigure{Two Hollyhock Leaves}{hollyhock}{Hollyhock, 1470.\sKamonNoJiten{7}
  }
  \subfigure{Three Bamboo Poles}{bamboo}{Three bamboo poles, 1470\sKamonNoJiten{7}
  }
  \subfigure{Plum blossom flowerpot}{plum}{ Plum blossom, Tsutsui Junkei, 1582\sSamuraiHeraldry{6.4}
  }
  \end{subfigs}
  \begin{subfigs}
  \jpgfigure{Bellflower}{bellflower}{Bellflower\sKenmonShokamon{7}}
    \subfigure{Wood Sorrel}{sorrel}{ Wood Sorrel\sOUmajirushi{1.32}}
    \jpgfigure{Apricot Leaf}{apricot-leaf}{Apricot Leaf, 1470\sKenmonShokamon{15}}
  \end{subfigs}
  \end{figure}

\section{Charges}

  What was depicted in period mon?  Stylizations of \textbf{plants} are the
  most common, used very widely.  Plant depictions can be tweaked in
  various ways by varying the position of the plant or the number of
  petals, buds, or leaves.  Common examples of plants include:

\begin{multicols}{2}
  \begin{itemize}
\item Chrysanthemum, the 16-petal version of which was
  used by the Imperial line (\emph{figures \ref{chrys}, \ref{3chrys}})
\item Paulownia, used by the Ashikaga clan who ruled Japan in the \MuromachiPeriod{} (\emph{figure \ref{paul}})
\item Hollyhock (\emph{figure \ref{hollyhock}})
\item Bamboo (\emph{figure \ref{bamboo}})
\item Plum blossom (\emph{figure \ref{plum}})
\item Chinese bellflower (\emph{figure \ref{bellflower}})
\item Wisteria
\item Wood sorrel (\emph{figure \ref{sorrel}})
\item Apricot leaf (\emph{figure \ref{apricot-leaf}})
  \end{itemize}
\end{multicols}

  \textbf{Simple geometric shapes} were also common.  Shapes were given
  various names based on objects they resemble.  Common examples
  include:

\begin{multicols}{2}
\begin{itemize}
\item A solid circle, or ``sun disc" (\emph{figure
  \ref{sundisc}}).  This is familiar today as the flag of modern Japan,
  but it was not used as a national emblem in period.\sSamuraiHeraldry{52}
\item Clusters of circles, called ``stars". (\emph{figure \ref{stars}})
\item A hollow circle, or ``snake eye", originally patterned after a leather
  bow-string spool\sSamuraiHeraldry{G2}\sFCoJ{94} (\emph{figure \ref{snake}})
\item A circle with lines, or ``rice bowl" (\emph{figure \ref{rice}})
  % TODO: the nail puller is the angled one, this is not \sFCoJ{100}
\item A hollow square (\emph{figure \ref{square}})
\item Several such squares, or ``eyes"\sFCoJ{113} (\emph{figure \ref{eyes}})
\item Several solid squares, or ``stones"\sKenmonShokamon{32} (\emph{figure \ref{stone}})
\item Three joined rhombuses,\sSamuraiHeraldry{63} called ``chestnut" or ``pine bark" depending on proportions (\emph{figures \ref{chestnut}, \ref{bark}})
\end{itemize}
\end{multicols}

  \begin{figure}[b!]
  \begin{subfigs}
  \subfigure{Sun Disc Banner}{sundisc}{\sOUmajirushi{1.11}
  }
  \subfigure{Shadowed, Nine Stars}{stars}{ Emperor Go-Shirakawa, 1160\sSamuraiHeraldry{4}
  }
  \subfigure{Snake Eye}{snake}{\sOUmajirushi{3.30}
  }
  \subfigure{Rice Bowl}{rice}{\sOUmajirushi{3.23}
  }
  \subfigure{Square}{square}{\sOUmajirushi{3.21}
  }
  \subfigure{Eyes}{eyes}{\sOUmajirushi{3.12}}
  \end{subfigs}
  \begin{subfigs}
  \subfigure{Early Stone}{stone}{\sKenmonShokamon{32}}
  \subfigure{Chestnut}{chestnut}{Ogasawara\sOUmajirushi{3.1}}
  \subfigure{Gourds with Rhombuses}{bark}{ ``Pine bark" diamonds on gourds, 1470\sKamonNoJiten{7}
  }
  \end{subfigs}
  \end{figure}

\clearpage

  \textbf{Kanji}, Japanese characters of Chinese origin, were another
  common choice.  These were often a reference to a patron deity, the
  first character of a clan's family name, or something that described
  the samurai in question.  One example is Koide Yoshichika, who used
  the first character in his family name, `ko', as his mon (\emph{figure
  \ref{ko}}).\sSamuraiHeraldry{19} Longer texts were sometimes also used
  (\emph{figure \ref{iroha}}).

  \begin{figure}
  \begin{subfigs}
  \subfigure{Ko}{ko}{\sOUmajirushi{5.8}}
  \jpgfigure{Well Character}{well}{1470\sKenmonShokamon{24}}
  \jpgfigure{Five and Notched Shapes}{notched}{1470\sKenmonShokamon{36}}
  \jpgfigure{Kenmon Turtle Shell Complex}{kcomplex}{1470\sKenmonShokamon{24}}
  \subfigure{Iroha Banner}{iroha}{\sOUmajirushi{3.32}}
  \end{subfigs}
  \begin{subfigs}
  \subfigure{Three Tomoe Circle}{tomoe}{Three tomoe.\sSamuraiHeraldry{J9}
  }
  \subfigure{Gate}{gate}{\sOUmajirushi{3.14}
  }
  \subfigure{Treasure Wheel}{treasure-wheel}{\sKenmonShokamon{27}
  }
  \subfigure{Swastika Banner}{swastika}{\sOUmajirushi{2.30}
  }
  \subfigure{Cross Banner}{cross}{\sOUmajirushi{3.19}
  }
  \end{subfigs}
  \end{figure}
  
  \textbf{Religion} was important to many samurai, and religious symbols
  beyond characters that refer to deities were commonly used.  Examples
  include:
  
\begin{itemize}
\item The comma shape of the tomoe, generally seen in threes (\emph{figure
  \ref{tomoe}}).  The tomoe has various
  interpretations focusing on Shintō and imperial connections.  It can
  be interpreted as a jewel or whirlpool, but it is most commonly
  associated with Hachiman, god of archery and war.
\item Shint\=o shrine gates (\emph{figure \ref{gate}})
\item Shrine amulets
\item The Buddhist treasure wheel (\emph{figure \ref{treasure-wheel}})
\item The swastika (\emph{figure \ref{swastika}}), 
  disallowed in the SCA, represents eternity and is strongly
  associated in Japan with Buddhism.
\end{itemize}

  After the introduction of Christianity with the arrival of the
  Portuguese in 1543, some Japanese chose \textbf{Christian symbols} such
  as crosses for their mon (\emph{figure \ref{cross}});\sSamuraiHeraldry{47}
   such symbolism was used openly until Christianity's violent
  suppression, which culminated in the massacre ending the Shimbara
  rebellion in 1638.

  Some mon incorporated \textbf{ordinary objects}, which often had symbolic
  significance.  Examples include:

\begin{multicols}{2}
\begin{itemize}
\item Arrow fletchings (\emph{figure \ref{fletching}})
\item Coins (\emph{figure \ref{coin}})
\item Fans (\emph{figure \ref{fan}})
\item Cart wheels (\emph{figure \ref{wheel}})
\item Paper umbrellas (\emph{figure \ref{umbrella}})
\item Knots (\emph{figure \ref{knot}})
\item Other ordinary objects (e.g., \emph{figure \ref{teapot}})
\end{itemize}
\end{multicols}


  \begin{figure}[b!]
  \begin{subfigs}
  \subfigure{Fletching Banner}{fletching}{\sOUmajirushi{1.27}
  }
  \subfigure{Coin}{coin}{\sOUmajirushi{5.15}
  }
  \subfigure{Cypress Fan}{fan}{Cypress fan, 1470\sKamonNoJiten{7}}
  \jpgfigure{Sun Fan}{fan2}{1470\sKenmonShokamon{13}}
  \subfigure{Carriage Wheel}{wheel}{ ``Genji" wheel, 1185\sDowerCrests{130}
  }
  \subfigure{Umbrella}{umbrella}{ Umbrella, 1470\sKenmonShokamon{34}
  }
  \subfigure{Other Knot}{knot}{ Knot, 1470\sKenmonShokamon{43}
  }
  \end{subfigs}
  \begin{subfigs}
    \subfigure{Teapot}{teapot}{ ``One" and ``great" teapot, 1470\sKenmonShokamon{43}}
    \subfigure{Eight-rung Ladder}{ladder}{1638\sSamuraiHeraldry{63}}
  \end{subfigs}
  \end{figure}

\clearpage

  \textbf{Landscape features} were also used.  Examples include:

\begin{itemize}
\item A wave (\emph{figure \ref{wave}})
\item Mount Fuji (\emph{figure \ref{fuji}})
\item ``Suhama", a stylized beach or inlet (\emph{figure \ref{suhama}})
\item A full moon with cloud (\emph{figure \ref{moon-with-cloud}})
\item A garden scene (\emph{figure \ref{garden}})
\item A landscape painted on a fan (\emph{figure \ref{fan-landscape}})
\end{itemize}

  Finally, the last category of charge is \textbf{animals}, which were used
  relatively rarely.  The most common animals were butterflies
  (\emph{figure \ref{butterfly}}) and various birds (e.g., \emph{figures
  \ref{hawk}, \ref{goose}, \ref{crane}, \ref{phoenix}}), but some other examples
  exist, including a horse (\emph{figure \ref{horse}}), a shrimp
  (\emph{figure \ref{shrimp}}), a lion (\emph{figure \ref{lion}}), a crab,
  and a rabbit.

  \begin{figure}
  \begin{subfigs}
  \subfigure[60]{Wave}{wave}{Saito Dosan, 1556\sSamuraiHeraldry{57}
  }
  \subfigure{Fuji}{fuji}{\sOUmajirushi{6.34}
  }
  \subfigure{Suhama}{suhama}{1470\sKenmonShokamon{48}
  }
  \subfigure{Moon With Cloud}{moon-with-cloud}{1470\sKenmonShokamon{57}
  }
  \subfigure{Garden}{garden}{1470\sKenmonShokamon{68}
  }
  \subfigure{Landscape On Fan}{fan-landscape}{1470\sKenmonShokamon{5}
  }
  \end{subfigs}
  \end{figure}

  \begin{figure}
  \begin{subfigs}
  \subfigure{Taira Butterfly}{butterfly}{ The Taira Butterfly, 1185\sSamuraiHeraldry{A4}
  }
  \subfigure{Hawk On Perch}{hawk}{Hawk, 1470\sKamonNoJiten{7}
  }
  \jpgfigure{Enclosed Goose}{goose}{Goose on a stirrup or in tongs, 1470\sKamonNoJiten{7}
  }
  \subfigure{Early Crane}{crane}{Crane, 1470\sKenmonShokamon{32}}
  \jpgfigure{Bands and Phoenix}{phoenix}{H\=o\=o, or Phoenix, 1470\sKenmonShokamon{17}}
  \jpgfigure{Mandarin Duck and Dove}{duck-and-dove}{A mandarin duck and a dove, 1470\sKenmonShokamon{28}}
  \end{subfigs}
  \begin{subfigs}
  \subfigure{Horse}{horse}{Horse, 1470\sKenmonShokamon{24}
  }
  \subfigure{Shrimp}{shrimp}{Shrimp, 1470\sKenmonShokamon{49}
  }
  \subfigure{Lion}{lion}{Lion with peonies, 1470\sKenmonShokamon{50}}
  \jpgfigure{Three Shells}{shells}{Shells, 1470\sKenmonShokamon{30}}
  \end{subfigs}
  \end{figure}

\section{Variation}

  \textbf{Most mon featured geometric symmetry}, either reflectional or
  rotational, whether composed of one charge or several.  Unlike in
  European heraldry, which often featured multiple instances of an asymmetric
  charge, such as an animal, all facing the same direction, most Japanese
  charges were symmetrical, and when an asymmetrical charge like
  a bird or butterfly was repeated in a mon, the charges would often face each other as
  a mirror image (\emph{figure \ref{facing-butterflies}}).

  \begin{figure}
  \begin{subfigs}
  \jpgfigure{Facing Butterflies}{facing-butterflies}{1470\sKenmonShokamon{10}}
  \jpgfigure{Three Stars}{three-stars}{1470\sKenmonShokamon{17}}
  \subfigure[50]{Five Feathers}{feathers}{1470\sKamonNoJiten{7}}
  \subfigure{Two And A Half Feathers}{half}{Kikuchi Taketoki, 1334\sSamuraiHeraldry{14}}
  \jpgfigure{Facing Paulownia in Ring}{facing-paulownia}{1470\sKenmonShokamon{14}}
  \subfigure{Three Fans}{three-fans}{1638\sSamuraiHeraldry{63}}
  \jpgfigure{Tomoe Triangle}{tomoe-triangle}{1470\sKenmonShokamon{21}}
  \end{subfigs}
  \end{figure}



  The practices for combining multiple instances of a charge into a
  mon were relatively restricted.  \textbf{Repeated charges within a mon tended to be
  standard charges}, with geometric shapes (\emph{figure \ref{three-stars}}), feathers (\emph{figures \ref{feathers}, \ref{half}}), or plants (\emph{figure \ref{facing-paulownia}}) the most
  common.\sSamuraiHeraldry{D2,J1} The arrangements tended to be rather
  simple.

  \clearpage

  \begin{figure}
  \begin{subfigs}
  \subfigure{Leaves In Ring}{ring}{\sOUmajirushi{4.32}
  }
  \subfigure{Three in Octogon}{octo}{Inaba Masanari, 1570\sSamuraiHeraldry{54 }
  }
  \subfigure{Turtle Shell Chrys}{turtle}{1470\sKenmonShokamon{22 }
  }
  \jpgfigure{Mulberry and One in Ring}{ringplus}{1470\sKenmonShokamon{28}}
  \jpgfigure{Turtle Shell Complex}{complex}{1573\sAshikagaBukan{17}}
  \jpgfigure{Suhama Bell}{bell}{1470\sKenmonShokamon{20}}
  \jpgfigure{Sparrows in Bamboo}{sparrows}{1470\sKenmonShokamon{34}}
  \end{subfigs}
  \end{figure}

  Relative to European heraldry, the ways that distinct elements were
  combined into a single mon were limited, and some more complicated
  means of variation were uncommon or unknown in period.
  While sometimes different elements were combined, only harmonious
  combinations were used, in contrast to European arms combining
  unrelated charges.\sDowerCrests{11} \textbf{The most common combination
  of elements was the addition of an enclosure} around a
  charge.\sDowerCrests{15}  This was most frequently a
  circular ring (\emph{figure \ref{ring}}) or other geometric 
  outline (\emph{figure \ref{octo}}).  
  A more complicated enclosure was a ``turtle shell", a hexagonal border
  with another thin hexagon inside (\emph{figure \ref{turtle}}).
  Occasionally, this enclosure could enclose
  multiple types of charges (\emph{figure \ref{ringplus}, \ref{complex}}).
  More elaborate enclosures were also known (e.g., \emph{figures
    \ref{bark}, \ref{goose}, \ref{bell}, \ref{sparrows}}).

  \begin{figure}
  \begin{subfigs}
  \subfigure{Pierced Bellflower}{pierce}{Akechi Mitsuhide, 1582\sSamuraiHeraldry{57}}
  \jpgfigure{Sun and Moon}{sun-moon}{1470\sKenmonShokamon{19}}
  \subfigure{Eye Yoshi Banner}{yoshi}{Takezaki Suenaga, 1274\sSamuraiHeraldry{57 }
  }
  \subfigure{Dai Wisteria}{dai}{Ōkubo Tadayo, 1593\sSamuraiSourcebook{73.4}
  }
  \subfigure{Peace Paulownia}{peace}{1470\sKamonNoJiten{7}
  }
  \subfigure{Pines And Three}{pines}{1470\sKenmonShokamon{31}
  }
  \end{subfigs}
  \end{figure}

  A similar type of variation was to place a charge on a disc, with
  the charge color and background matching and the disc contrasting,
  effectively `piercing' the disc with a charge (\emph{figure
  \ref{pierce}}).\sSamuraiHeraldry{57} Other geometric shapes could occasionally
  be used (e.g., \emph{figure ref{sun-moon}}).

  Characters were sometimes placed among other charges; for example,
  Takezaki Suenaga, when fighting the mongols in 1274, used three
  square `eye' designs and a character `yoshi', meaning ``good
  fortune" (\emph{figure \ref{yoshi}}).\sSamuraiHeraldry{12} Wisteria's depiction with two
  branches making a circle created an obvious place for another
  charge, which \=Okubo Tadayo (1531--1593) used to include the
  character `dai', meaning ``great" (\emph{figure \ref{dai}}).\sSamuraiSourcebook{73.4}
  An earlier samurai similarly placed the character `yasu', meaning ``peace",
  in a paulownia (\emph{figure \ref{peace}}).\sKamonNoJiten{7}
  More complicated arrangements were also possible,
  such as this mon with the character for ``three" and
  three multi-level pine trees (\emph{figure \ref{pines}}).
  
  Aside from enclosures and characters,  it was relatively
  uncommon for multiple 
  unrelated charges to be combined.
  Different charges could be arranged vertically (\emph{figure \ref{in-pale}})
    or, less commonly,
  horizontally (\emph{figure \ref{arrowspine}}).
  Another arrangment was showing a charge emerging from another charge,
  most famously the mon used by Kusunoki Masashige (1294--1336),
  who symbolized his support for the emperor by depicting the imperial
  chrysanthemum supported by water (\emph{figure \ref{water}}).\sSamuraiHeraldry{B1} Another form of emergence was a repeated design, most commonly a plant,
  emerging from the inside of a ring (e.g., \emph{figures \ref{plant-ring},
  \ref{weird-ring}}).
  Examples of charges placed on a background design
  are found as well (e.g., \emph{figure \ref{road-tomoe}}).
  A one-and-two arrangement with different charges is occasionally found;
  Yamana S\=ozen Mochitoyo (1404--1473) used a paulownia above two
  gentian-like plants as his mon (\emph{figure \ref{twoplants}}).\sSamuraiSourcebook{73.14} (See also \emph{figures \ref{luck-circles}, \ref{diff1and2}}.)

  \begin{figure}[b]
  \begin{subfigs}
  \subfigure{One Three Stars Banner}{in-pale}{
    \sOUmajirushi{1.18}
  }
  \subfigure{Arrows And Pine}{arrowspine}{1470\sKenmonShokamon{28}}
  \subfigure{Floating Chrysanthemum}{water}{Kusunoki Masashige, 1336\sSamuraiSourcebook{22.3}}
  \jpgfigure{Three in Ring with Flowers}{plant-ring}{1470\sKenmonShokamon{33}}
  \jpgfigure{Crossed Feathers in Ring}{weird-ring}{1470\sKenmonShokamon{11}}
  \jpgfigure{Road Tomoe}{road-tomoe}{1470\sKenmonShokamon{18}}
\end{subfigs}
\begin{subfigs}
  \subfigure{Two Plants}{twoplants}{Yamana Mochitoyo, 1473\sSamuraiSourcebook{73.14}}
  \jpgfigure{Luck and Circles}{luck-circles}{1470\sKenmonShokamon{15}}
  \jpgfigure{Melons and Bell}{diff1and2}{1470\sKenmonShokamon{26}}
  \end{subfigs}
\end{figure}

\clearpage

\begin{figure}[t]
  \begin{subfigs}
  \subfigure{Wisteria Tomoe}{wisteria-tomoe}{\sOUmajirushi{3.9}
  }
  \subfigure{Tomoe Stars}{tomoe-stars}{Tomoe stars\sOUmajirushi{4.31}}
  \jpgfigure{Coin and Stars}{coin-stars}{A coin with stars, 1470\sKenmonShokamon{15}}
  \jpgfigure{Flower Rhombus}{no-blades}{1470\sKenmonShokamon{9}}
  \jpgfigure{Bladed Flower Square}{blades}{1470\sKenmonShokamon{22}}
  \end{subfigs}
  \end{figure}

  Another form of variation was depicting one element in such a way
  that it mimiced the standard depiction of another element.  This
  style was rare before the \EdoPeriod, but one example of mimicry is
  the mon of Kuroda Yoshitaka (1546--1604), which depicted wisteria in
  the shape resembling three tomoe swirls (\emph{figure
  \ref{wisteria-tomoe}}).\sSamuraiSourcebook{36.9,53} Variations on
  the standard nine stars arrangement were common,  replacing
  all the stars (\emph{figure
  \ref{tomoe-stars}}) or just the middle one (\emph{figure \ref{coin-stars}}).

  In addition to these obvious forms of variation, there were more
  subtle variations.  Different branches of the same family would
  sometimes use closely-related mon that differed only by minor
  details.  Similarly, the limited set of motifs from which crests
  were most commonly drawn meant that at times completely unrelated
  families would end up with almost identical mon by
  coincidence.\sDowerCrests{10} Since there was no formal
  registration, exactly what constituted a different mon vs a
  different depiction of the same mon was unclear. Possibly the most notable
  minor charge variation is the addition of sword blades to flowers to
  create a more martial motif. Compare a flower square or rhombus without
  blades (\emph{figure \ref{no-blades}}) to the same motif with blades added (\emph{figure \ref{blades}}).


  \begin{figure}[b]
  \begin{subfigs}
  \subfigure{Bishamonten}{bisha}{The first character of Bishamon-ten, Guardian of the North; 1578\sSamuraiHeraldry{25}
  }
  \subfigure{Star And Crescent}{chiba}{1455\sSamuraiHeraldry{8}
  }
  \subfigure{Aoyama Banner}{aoyama}{Aoyama's ``blue mountain"\sOUmajirushi{2.15}
  }
  \end{subfigs}
  \end{figure}  

\section{How were mon chosen?}

  There were a number of ways one could choose a mon.  Some presumably
  just picked designs that they liked; at least, plenty have no
  obvious connection to their bearer.  Some picked designs that
  referred to a \textbf{patron deity} (\emph{figure \ref{bisha}}) or their
  \textbf{family name}, such as Koide Yoshichika's character-based mon,
  described above (\emph{figure \ref{ko}}).

  Some picked mon that relate to \textbf{stories about their ancestors}.
  For example, in the 14th century, the Chiba used a ``star and
  crescent" mon (\emph{figure \ref{chiba}}) and other star-based designs,
  which was inspired by a legend from 931, when the Chiba were on the
  brink of defeat.  The constellation of the Plough was bright in the
  night sky, so they prayed to Myomi Bosatsu, a deity associated with
  the Plough.  As a result, the deity appeared to them in a vision,
  and they then emerged victorious.\sSamuraiHeraldry{11} Other such
  stories about heraldry are common, though it's unclear if anyone
  used a story about themselves personally in their heraldry, rather
  than a story about an ancestor.\sSamuraiHeraldry{11}

  Japanese devices occasionally used a type of \textbf{pun} (similar to
  Western heraldry's ``canting") where part of the device refers like
  the person's name, beyond just using a character.  One example is Aoyama
  Tadanari (1551--1613), whose family name means ``blue mountain"; his
  mon featured a blue character for mountain (`yama') on a white
  background, or vice versa (\emph{figure
  \ref{aoyama}}).\sSamuraiHeraldry{32}\sOUmajirushi{2.15} (This
  also is an example of the significance of color in mon.)  A
  similar example is a Kuroda mon that used a black disc on white to
  refer to their name, which means ``black field".\sSamuraiSourcebook{53}
   Similarly, since many Japanese families are named after
  plants, some mon depicted a plant referred to in a clan
  name.\sDowerCrests{33}

  \textbf{Shrines, as could be expected, would often choose explicitly
  religious motifs for their mon}, with the tomoe comma design in
  particular being used by shrines dedicated to Hachiman, god of
  archery and war.  Other shrines would use plant-based motifs, either
  to show patronage, as with the paulownia mon granted to shrines by
  Toyotomi Hideyoshi (1537--1598) when he ruled Japan as Sh\=ogun, or
  due to symbolic connections, as with the sheaves of rice mon favored
  by shrines of Inari, god of rice.\sEncyclopediaOfShinto{Shinmon}

\section{Researching Period Mon}

  There are three types of period sources we can use to reliably research
  mon: written accounts, paintings (mainly of battles), and, most
  usefully, compendia.
  There are two main compendia that were published before 1650:
  \fnhref{http://dl.ndl.go.jp/info:ndljp/pid/2533035}{\emph{Kenmon Shokamon} (見聞諸家紋),} published 1467--1470, and 
  \emph{O-umajirushi} (御馬印),\footnote{I have an annotated
  translation available at \url{http://o-umajirushi.xavid.us/}; the original
  is available at \url{http://dl.ndl.go.jp/info:ndljp/pid/1288484?__lang=en}.} published 1624--1644.  Both of these compendia
  record the devices used in the preceding periods of conflict. They contain
  a wide variety of mon, providing many of the figures in this handout.
  I've also uploaded a scanned copy of \fnhref{http://fireflies.xavid.us/wp-content/uploads/2015/11/Ohatamoto-S\%C5\%8Dshirushizu.pdf}{\emph{Ohatamoto S\=oshirushizu}
  (御簱本惣印図; ``Shogunal Vassals All Emblem Drawings''),} a book from circa 1634 available
  at the Library of Congress.

  Secondary sources that include period mon include the
  \fnhref{http://dl.ndl.go.jp/info:ndljp/pid/1015270}{\emph{Daibukan} (大武鑑;
  ``Great Book of Heraldry''), which is available from the NDL};
  through the right half of page 55, as numbered there, is period. It is a
  compilation of earlier secondary sources, including the
  \fnhref{http://www.wul.waseda.ac.jp/kotenseki/html/bunko01/bunko01_01827/index.html}{\emph{Kamakura Bukan}
  (鎌倉武鑑),} covering 1185--1333, and the \fnhref{http://www.wul.waseda.ac.jp/kotenseki/html/bunko20/bunko20_00368_0007/index.html}{\emph{Ashikaga-ke Bukan}
  (足利家武鑑),} covering 1336--1573 (sharing many mon with \emph{Kenmon
    Shokamon}).

  There are also modern compendia of mon, but ones in English and most
  of the ones in Japanese lack specific dates and citations and focus
  mainly on mon as they were standardized later in the \EdoPeriod{} or
  as they're used today, and thus aren't good resources for period
  practice.  The compendia cited at the end of this handout have at least
  some information about when motifs were used and thus can be somewhat
  useful.

\section{Battlefield Heraldry Beyond Banners}

Mon were not only used on banners on the battlefield. They might also
be used on curtains around a camp (e.g., \emph{figure \ref{curtain}})
or on surcoats called jinbaori worn over armor (e.g., \emph{figure
  \ref{surcoat}}).

In addition, while mon are the type of Japanese heraldry most directly
comparable to European coats of arms and badges, there were other
types of heraldry that were just as important for battlefield
identification. Helmets and three-dimensional objects were recorded in
heraldic compendia like \emph{O-umajirushi} alongside banners.

Samurai would sometimes wear distinctive identifying helmets that could
be quite elaborate. While the more extreme ones like in figure
\ref{r-helm} were probably only worn in processions, not in battle, they
may have been carried on poles for heraldic purposes in battle, Simpler
identifying helmets were worn in battle.

Large three-dimensional objects of various types were often used on
battlefields as standards and devices, having the advantage of being
easily visible from a distance regardless of angle or wind
conditions. These might be representations of everyday objects (e.g.,
\emph{figure \ref{hat-standard}}) or more abstract designs (e.g.,
\emph{figure \ref{hoop-standard}}). The most commonly used of these
are the fukinuki (e.g., \emph{figure \ref{fukinuki}}), a sort of wind
sock, and the horo (e.g., \emph{figure \ref{horo}}), a canopy on a
framework designed to inflate as its bearer rides.

\begin{figure}[b!]
\begin{subfigs}
  \jpgfigure[150]{Tokugawa Curtain}{curtain}{1634\sOhatamoto{3}}
  \jpgfigure[150]{Eight Stars Jinbaori}{surcoat}{1634\sOhatamoto{15}}
  \jpgfigure[150]{Big Ear Helm}{r-helm}{1600\sRijksH{}}
  \jpgfigure[150]{Cloud-Horned Helmet}{cloud-helm}{1643\sOUmajirushi{2.13}}
\end{subfigs}
\begin{subfigs}
  \jpgfigure[125]{Hat Standard}{hat-standard}{1644\sOUmajirushi{2.21}}
  \jpgfigure[125]{Hoop Standard}{hoop-standard}{1644\sOUmajirushi{2.15}}
  \jpgfigure[75]{White Fukinuki}{fukinuki}{1644\sOUmajirushi{2.5}}
  \jpgfigure[125]{Black Horo}{horo}{1644\sOUmajirushi{2.5}}
\end{subfigs}
\end{figure}

\section{Mon in the SCA}

  While mon may be registered as devices or badges in the SCA, and
  many mon-style devices are currently registered, the SCA heraldic
  rules default to a generic European standard, and thus it is not
  always obvious how to register authentic mon in the SCA.

\subsection{Ways to Register Mon}

  First, under SENA (the ``new rules"), there are two main ways to
  register a device, and mon can be registered under either.  The
  common way of registering devices are the Core Style Rules 
  (\fnhref{http://heraldry.sca.org/sena.html\#A1A1}{SENA A.1.A.1}), based on
  the heraldry of Western Europe.  \textbf{Devices that are not registrable
  under Core Style but follow period examples can still be registered
  under the Individually Attested Patterns rules}
  (\fnhref{http://heraldry.sca.org/sena.html\#A4}{SENA A.4}).  You can also
  think of these two methods as the ``easy" way and the ``hard" way.

  \textbf{Under the Core Style Rules, you can register any device that fits
  into the framework of the default SCA heraldic rules, with at most
  one element from outside the European default}, called a Step from
  Period Practice (SFPP).  While these are optimized for European
  usage, since most mon are simple and use a single charge, you can
  often have that charge be your SFPP and otherwise have a device
  compatible with European standards.  This method of registration
  doesn't require you to have extensive documentation, which makes it
  relatively straightforward.  You do generally need to provide
  evidence that your SFPP is an element used in non-European armory or
  a plant or animal known to pre-1600 Europe, but unlike when using
  the Individually Attested Patterns rules, a single example is
  sufficient.

  SENA recognizes that not all period devices fit under the default
  rules, and also allows registration using \textbf{Individually Attested
  Patterns}. When using these rules, all elements used, both charges
  and arrangement, must be from the same general time and place, and
  the device must be still expressible in blazon, the language of
  Western heraldry.  \textbf{For every element not in the core style rules,
  either three closely matching examples or six examples of comparable
  complexity must be included.  All examples must be pre-1650 examples
  from different families.}  While many books of mon are readily
  available, they generally contain many post-1650 mon and often do
  not contain usage dates, so finding good examples can be a
  considerable amount of work.

\subsection{Mon vs Devices}

  A device, in the SCA, is a heraldic design that represents a
  particular individual or group.  While a mon could easily be used as
  a Japanese-style device, a single mon is not the only design that
  could be considered a ``device".  Since different members of a clan
  or different divisions of an army might use the same mon but
  distinct banners, you could consider a ``device" an overall banner
  design.  As discussed above, samurai banners might have multiple mon
  or other graphical elements in addition to the mon, or even no mon
  at all.

  We have examples of banners with a variety of designs, including:
  
\begin{itemize}
\item 2, 3, or 5
  copies of the same mon in the same color on a solid background
  (\emph{figures \ref{3x}, \ref{5x}})
\item two different mon, possibly in
  different colors, on a solid background (\emph{figure \ref{diff}})
\item two copies of the same mon in different colors on a divided
  background
\item mon with text passages (often slogans or prayers)
\item mon with simple or complex-line stripes (\emph{figures \ref{bigstripe}, \ref{twostripes},
  \ref{manystripes}, \ref{diagonal}}) (Stripes
  were a common means for distinguishing units of the same
  army.\sSamuraiHeraldry{24})
\item mon with background changes (\emph{figure \ref{change}})
\end{itemize}

  Devices of this sort could be registered in the same way as single
  mon designs.  Note, however, that using multiple Japanese elements
  may make a device no longer fall under the Core Style Rules,
  requiring Individually Attested Pattern examples for registration.

  \begin{figure}
  \begin{subfigs}
  \subfigure[100]{Three Star Grid Banner}{3x}{\sOUmajirushi{1.14}
  }
  \subfigure[100]{Five Chestnut Banner}{5x}{\sOUmajirushi{3.1}
  }
  \subfigure[100]{Sorel Hexagon Banner}{diff}{\sOUmajirushi{6.29}
  }
  \subfigure[100]{Hon Stripe Banner}{bigstripe}{\sOUmajirushi{1.24}
  }
  \subfigure[100]{Crane Stripes Banner}{twostripes}{\sOUmajirushi{2.16}
  }
  \subfigure[100]{Stripey Box Banner}{manystripes}{\sOUmajirushi{1.12}
  }
  \subfigure[100]{Diagonal Plum Banner}{diagonal}{\sOUmajirushi{2.9}
  }
  \subfigure[100]{Seven Stars Change Banner}{change}{\sOUmajirushi{2.32}
  }
  \end{subfigs}
  \end{figure}

\subsection{Color}

  The modern view of mon as color-free designs has lead to some
  confusion with mon registration in the SCA, since SCA devices always
  include color.  However, in period practice, mon were depicted on
  banners with specific contrasting foreground and background colors.
  Thus, registering mon using two specific contrasting colors reflects
  how mon were used for identification in period.

  Mon can be used with different color combinations, either for
  different uses by the same individual or by different individuals.
  For SCA use, it might make sense to register multiple color
  variations for different members of the same clan.

  Conveniently, the colors used in Japanese heraldry, as discussed
  above, were generally the same set used in Western heraldry: mostly
  the five ``lucky colors" of yellow (or gold), blue, red, white, and
  black, plus occasionally purple.  In terms of color combinations,
  while many banners followed the SCA-standard ``rule of tincture"
  that defines what colors contrast, some did not: black on red and
  red on black are common\sSamuraiHeraldry{C}, and I have also seen
  red on blue\sSamuraiHeraldry{C4} and black on
  green\sSamuraiHeraldry{H13}.  Registrations of these color schemes
  would need to use the Individually Attested Patterns rules.

  It should be noted that an individual might not always use the
  colors of their banner when displaying their mon on other items,
  such as armor or clothing, so authentic usage would not require
  using your registered colors in all cases.

\subsection{Restricted Charges}

  Since there was no formal registration of mon in the SCA period,
  samurai could generally pick any mon they wished.  There were a few motifs
  that were, however, effectively reserved in period.  The most well
  known is the sixteen-petaled chrysanthemum (``kiku") mon used
  by the Emperor (\emph{figure \ref{chrys}}), with variations used by other members of the
  imperial family.\sSamuraiHeraldry{6} While people
  outside the imperial line, such Kusunoki Masashige (\emph{figure \ref{water}}), did use a
  chrysanthemum variation, this was an Imperial gift\sSamuraiHeraldry{13},
  so such usage in the SCA might be presumptuous.

  A similar restriction was in place for the paulownia (``kiri") mon
  given to the Ashikaga clan by Emperor Godaigo when they effectively
  ruled Japan (\emph{figure \ref{paul}}).  While the Ashikaga bestowed the right to used the
  paulownia to powerful supporters, the motif is strongly associated
  in period with imperial and shōgunal favor, and thus its usage might
  also be presumptuous.\sDowerCrests{68}

  The swastika, while a Buddhist symbol in Japan used in
  period mon, is forbidden in the SCA for obvious reasons.

\subsection{Other Relevant Rules}

  \textbf{While many historical Japanese banners consisted solely of a single
  Japanese character or a longer Japanese phrase or sentence, such
  devices cannot be registered in the SCA}, as ``their registration
  might limit someone from using their initials or a written version
  of their name or motto'' (see
  \fnhref{http://heraldry.sca.org/laurel/sena.html\#A3E3}{SENA A.3.E.3}.)
  Furthermore, even when Japanese characters are combined with other
  elements, such as the rings and enclosures found on historical
  banners, \textbf{all characters, in any alphabet, are considered identical
  for purposes of conflict checking}. (See ``Constantina von Ravenna''
  in the \fnhref{http://heraldry.sca.org/loar/2008/05/08-05lar.html\#141}{May
  2008 LoAR}.)  This means that Japanese mon that use different
  characters will often conflict, especially given the simplicity of
  authentic mon.  This issue could potentially be reduced if those who
  register such devices submit Blanket Permission to Conflict for
  devices that do not use the identical character.  (See 
  \fnhref{http://heraldry.sca.org/admin.html\#III.C}{III.C 
  in the Administrative Handbook}.)

\section{Meta}

  I am Kih\=o, a wandering poet who presently abides in the Barony of
  Carolingia.  I welcome questions, feedback, corrections, and requests for
  assistance assembling sources for use with the Individually Attested
  Patterns Rules.  I may be contacted
  at \texttt{kihou@mit.edu}.  I have a blog where I discuss
  period Japan, \url{http://fireflies.xavid.us/}, and a database where I'm
  gradually compiling period Japanese heraldry from various sources,
  \url{http://mon.xavid.us/}.

  The most up-to-date version of this handout can be found at\\
  \url{http://mon.xavid.us/Introduction\%20to\%20Japanese\%20Crests.pdf}.

  Feel free to use, modify, or distribute this handout under the terms
  of \fnhref{http://creativecommons.org/licenses/by-sa/3.0/}{Creative Commons Attribution-ShareAlike 3.0 Unported}.  Basically, you
  can distribute it freely as long as you give me credit and allow
  others to distribute any changes you make under the same terms.
  (The mon images themselves are, of course, in the Public Domain.)

  Research for this class was conducted at the Library of Congress and was facilitated by the staff of the Asian Reading Room.

\section{Sources Cited}

\begin{description}
\cited
\end{description}

\end{document}
